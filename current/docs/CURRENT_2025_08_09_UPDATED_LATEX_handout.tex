\documentclass[11pt,letterpaper]{article}
\usepackage[margin=1in]{geometry}
\usepackage{graphicx}
\usepackage{hyperref}
\usepackage{enumitem}
\usepackage{amsmath}
\usepackage{amsfonts}
\usepackage{booktabs}
\usepackage{xcolor}
\usepackage{fancyhdr}
\usepackage{titlesec}
\usepackage{url}
\usepackage{float}

% Configure hyperlinks
\hypersetup{
    colorlinks=true,
    linkcolor=blue,
    urlcolor=blue,
    citecolor=blue
}

% Configure headers
\pagestyle{fancy}
\fancyhf{}
\rhead{Jiang | Immigrants Against Immigrants?}
\lhead{ASA 2025}
\cfoot{\thepage}

% Configure section formatting
\titleformat{\section}{\Large\bfseries}{\thesection}{1em}{}
\titleformat{\subsection}{\large\bfseries}{\thesubsection}{1em}{}
\titleformat{\subsubsection}{\normalsize\bfseries}{\thesubsubsection}{1em}{}

% CUSTOM SMALLER BULLETS
% Define smaller bullet symbols
\newcommand{\smallbullet}{\raisebox{0.25ex}{\scalebox{0.6}{$\bullet$}}}
\newcommand{\smalldash}{\raisebox{0.1ex}{\scalebox{0.8}{--}}}
\newcommand{\smallcirc}{\raisebox{0.25ex}{\scalebox{0.6}{$\circ$}}}
\newcommand{\smalldot}{\raisebox{0.25ex}{\scalebox{0.5}{$\cdot$}}}

% CUSTOMIZABLE INDENTATION SYSTEM
% Define your indentation step size here:
\newlength{\indentStep}
\setlength{\indentStep}{1.5em}  % Change this value to adjust indentation spacing

% Calculate indentations automatically
\newlength{\levelOneIndent}
\newlength{\levelTwoIndent}
\newlength{\levelThreeIndent}
\newlength{\levelFourIndent}

\setlength{\levelOneIndent}{2\indentStep}      % 3.0em with 1.5em step
\setlength{\levelTwoIndent}{3.5\indentStep}    % 5.25em with 1.5em step
\setlength{\levelThreeIndent}{4.5\indentStep}  % 6.75em with 1.5em step
\setlength{\levelFourIndent}{5.5\indentStep}   % 8.25em with 1.5em step

% Apply the calculated indentations with smaller bullets
\setlist[itemize,1]{
    leftmargin=\levelOneIndent,
    labelsep=1em,
    itemindent=0pt,
    label=\smallbullet,
    itemsep=2pt,
    parsep=0pt,
    topsep=3pt
}
\setlist[itemize,2]{
    leftmargin=\levelTwoIndent,
    labelsep=1em,
    itemindent=0pt,
    label=\smalldash,
    itemsep=1pt,
    parsep=0pt,
    topsep=2pt
}
\setlist[itemize,3]{
    leftmargin=\levelThreeIndent,
    labelsep=1em,
    itemindent=0pt,
    label=\smallcirc,
    itemsep=1pt,
    parsep=0pt,
    topsep=2pt
}
\setlist[itemize,4]{
    leftmargin=\levelFourIndent,
    labelsep=1em,
    itemindent=0pt,
    label=\smalldot,
    itemsep=1pt,
    parsep=0pt,
    topsep=2pt
}

% Enumerate with slightly more space for numbers
\setlist[enumerate,1]{
    leftmargin=\dimexpr\levelOneIndent+0.3em\relax,
    labelsep=1em,
    itemindent=0pt,
    itemsep=2pt,
    parsep=0pt,
    topsep=3pt
}
\setlist[enumerate,2]{
    leftmargin=\dimexpr\levelTwoIndent+0.3em\relax,
    labelsep=1em,
    itemindent=0pt,
    itemsep=1pt,
    parsep=0pt,
    topsep=2pt
}
\setlist[enumerate,3]{
    leftmargin=\dimexpr\levelThreeIndent+0.3em\relax,
    labelsep=1em,
    itemindent=0pt,
    itemsep=1pt,
    parsep=0pt,
    topsep=2pt
}
\setlist[enumerate,4]{
    leftmargin=\dimexpr\levelFourIndent+0.3em\relax,
    labelsep=1em,
    itemindent=0pt,
    itemsep=1pt,
    parsep=0pt,
    topsep=2pt
}

% Custom command for compact descriptions with proper indentation
\newcommand{\compactdesc}[2]{\item \textbf{#1:} #2}

% Define highlighting for key findings
\newcommand{\keyfinding}[1]{\colorbox{yellow!20}{\textbf{#1}}}
\newcommand{\surprise}[1]{\textcolor{red}{\textbf{#1}}}

\title{\textbf{Immigrants Against Immigrants?}\\
\large The Great Generational Reversal: Non-Linear Assimilation in Latino Immigration Attitudes (2002--2022)}

\author{Ann Jiang, UC San Diego\\
\href{mailto:annjiang@ucsd.edu}{annjiang@ucsd.edu} | \href{https://github.com/annjiangcodes}{github.com/annjiangcodes}}

\date{August 9, 2025}

\begin{document}

\maketitle

\section{Revolutionary Finding: The 2021-2022 Generational Role Reversal}

\subsection{The Shocking Discovery}

\keyfinding{Between 2021-2022, first-generation immigrants became MORE LIBERAL than second-generation Americans} -- a complete inversion of traditional assimilation theory expectations.

\begin{itemize}
    \compactdesc{2002 Baseline}{1st Gen = -0.30 (most restrictionist), 3rd+ Gen = +0.18 (most liberal)}
    \compactdesc{2022 Outcome}{1st Gen = +0.33 (most liberal), 2nd Gen = -0.15 (most restrictionist)}
    \compactdesc{Magnitude}{+0.33 liberalism shift in one year for first generation}
    \compactdesc{Theoretical Impact}{\surprise{Complete invalidation of linear assimilation models}}
\end{itemize}

\textbf{Central Question:} How do we explain this unprecedented generational role reversal that contradicts 100+ years of assimilation theory?

\section{The Multi-Track Generational Response Model}

Our 20-year analysis reveals three distinct generational ``tracks'' that respond to immigration policy environments through fundamentally different mechanisms.

\subsection{Track 1: Reactive Integration (First Generation)}

\textbf{Mechanism:} \textit{Policy threat → group consciousness → delayed but intense liberalization}

\begin{itemize}
    \compactdesc{Response Pattern}{2-3 year delayed reactions to policy changes}
    \compactdesc{2002-2008}{Pre-recession restrictionism (mean = -0.25)}
    \compactdesc{2009-2016}{Obama-era gradual liberalization (mean = -0.05)}
    \compactdesc{2017-2020}{Trump-era temporary retreat (mean = -0.18)}
    \compactdesc{2021-2022}{\keyfinding{Post-Trump reactive mobilization (mean = +0.13)}}
    \compactdesc{Volatility}{Highest attitude instability (CV = 3.35)}
\end{itemize}

\surprise{\textbf{Sociological Surprise:}} Recent immigrants don't become ``more American'' -- they become more pro-immigration through integration experiences.

\subsection{Track 2: Strategic Distancing (Second Generation)}

\textbf{Mechanism:} \textit{Status anxiety → identity differentiation → counter-cyclical positioning}

\begin{itemize}
    \compactdesc{Response Pattern}{Immediate, strategic responses to immigration salience}
    \compactdesc{Key Behavior}{When 1st generation liberalizes, 2nd generation restricts}
    \compactdesc{2016 Gap}{Largest divergence from 1st generation (-0.32 difference)}
    \compactdesc{2021-2022}{Sustained restrictionist consolidation despite Biden policies}
    \compactdesc{Legalization Decline}{\surprise{Fastest decline rate among all generations (-1.99 per decade)}}
\end{itemize}

\surprise{\textbf{Sociological Surprise:}} The ``in-between'' generation actively distances from immigrant identity rather than serving as a bridge.

\subsection{Track 3: Settled Americanism (Third+ Generation)}

\textbf{Mechanism:} \textit{Stable American identity → consistent liberal baseline → minimal policy responsiveness}

\begin{itemize}
    \compactdesc{Response Pattern}{Stable liberal baseline with minimal policy sensitivity}
    \compactdesc{Mean Liberalism}{+0.22 (consistently most liberal until 2022)}
    \compactdesc{Policy Responsiveness}{Lowest among all generations}
    \compactdesc{Volatility Pattern}{\surprise{Stable liberalism but extreme restrictionism volatility (CV = 17.8)}}
\end{itemize}

\section{Data \& Gold Standard Methodology}

\subsection{Enhanced Dataset}
\textbf{Pew Research Center's National Survey of Latinos (NSL) + Comprehensive Analysis}
\begin{itemize}
    \compactdesc{Temporal Scope}{20 years, 13 survey waves (2002--2022)}
    \compactdesc{Sample Power}{30,869 valid observations with generation labels}
    \compactdesc{Policy Coverage}{4 presidential administrations, 3 major immigration policy shifts}
    \compactdesc{Statistical Approach}{Weighted OLS without fixed effects (NOFE) + fine-grained temporal analysis}
\end{itemize}

\subsection{Three-Index Measurement System}
\begin{enumerate}
    \item \textbf{Liberalism Index} (59.7\% coverage)
        \begin{itemize}
            \item Components: Legalization support, DACA support, pro-immigrant sentiment
            \item Key Finding: First generation slope = +0.020/year (p < 0.05)
        \end{itemize}
        
    \item \textbf{Restrictionism Index} (62.7\% coverage)
        \begin{itemize}
            \item Components: Border security, deportation support, ``too many immigrants''
            \item Key Finding: Complex generational interactions over time
        \end{itemize}
        
    \item \textbf{Deportation Concern Index} (22\% coverage)
        \begin{itemize}
            \item Components: Personal deportation worry, family threat perception
            \item Key Finding: Tracks with first-generation policy responsiveness
        \end{itemize}
\end{enumerate}

\begin{figure}[H]
    \centering
    \includegraphics[width=0.95\textwidth]{../../outputs/CURRENT_2025_08_09_FIGURES_gold_standard/three_indices_by_generation_2002_2022.png}
    \caption{\textbf{The Great Generational Reversal:} Immigration attitudes by generation across three key indices, showing the dramatic 2021-2022 role reversal that challenges traditional assimilation theory.}
    \label{fig:three_indices}
\end{figure}

\section{Critical Temporal Inflection Points}

\subsection{2011: The Obama Deportation Peak}
\begin{itemize}
    \compactdesc{Policy Context}{Peak deportation enforcement (400,000+ annual deportations)}
    \compactdesc{Generational Response}{Maximum divergence across all measures}
    \compactdesc{First Generation}{Defensive restrictionist positioning}
    \compactdesc{Second Generation}{Strategic liberal positioning}
    \compactdesc{Lasting Impact}{Established generational response patterns}
\end{itemize}

\subsection{2016: Pre-Trump Anticipatory Positioning}
\begin{itemize}
    \compactdesc{Policy Context}{Trump campaign immigration focus, wall rhetoric}
    \compactdesc{Generational Gap}{Largest first-second generation difference (-0.32)}
    \compactdesc{Anticipatory Effects}{\surprise{Attitude shifts BEFORE policy implementation}}
    \compactdesc{Strategic Behavior}{Generations position for expected policy environment}
\end{itemize}

\subsection{2021-2022: Post-Trump Reactive Consolidation}
\begin{itemize}
    \compactdesc{Policy Context}{Biden administration, immigration reform discussion}
    \compactdesc{First Generation}{Massive liberalization (+0.33 shift)}
    \compactdesc{Second Generation}{Sustained restrictionist consolidation}
    \compactdesc{Theoretical Impact}{\keyfinding{Complete role reversal invalidates linear assimilation}}
\end{itemize}

\begin{figure}[H]
    \centering
    \includegraphics[width=0.95\textwidth]{../../outputs/CURRENT_2025_08_09_FIGURES_gold_standard/overall_trends_2002_2022.png}
    \caption{\textbf{Policy Period Effects:} Overall population trends mask the dramatic generational divergence occurring beneath aggregate stability.}
    \label{fig:overall_trends}
\end{figure}

\section{The Legalization Support Collapse: A Methodological Mystery}

\subsection{The Unprecedented Decline}
\surprise{\textbf{Shocking Finding:}} Legalization support collapsed from 80-95\% to 0\% across ALL generations by 2021-2022.

\begin{itemize}
    \compactdesc{First Generation}{95.9\% → 0\% (-1.52 per decade)}
    \compactdesc{Second Generation}{85.8\% → 0\% (-1.99 per decade, fastest)}
    \compactdesc{Third+ Generation}{79.3\% → 0\% (-1.74 per decade)}
    \compactdesc{Statistical Significance}{p < 0.01 for all generations}
\end{itemize}

\subsection{Potential Explanations}
\begin{enumerate}
    \item \textbf{Measurement Artifact:} Question wording or coding changes
    \item \textbf{Extreme Policy Response:} Trump-era backlash effects
    \item \textbf{Sample Composition:} Changes in who responds to immigration surveys
    \item \textbf{Definitional Shift:} What ``legalization'' means has changed
\end{enumerate}

\textbf{Research Priority:} This anomalous pattern requires immediate methodological investigation.

\section{Within-Generation Heterogeneity: The Complexity Hidden by Means}

\subsection{Distribution Analysis}
\begin{itemize}
    \compactdesc{Third+ Generation}{Highest heterogeneity (SD = 1.21) -- spans full political spectrum}
    \compactdesc{Second Generation}{Moderate heterogeneity (SD = 1.01) -- varied identity strategies}
    \compactdesc{First Generation}{Most homogeneous (SD = 0.84) -- shared immigrant experience}
\end{itemize}

\subsection{Volatility Patterns}
\surprise{\textbf{Bipolar Third+ Generation:}} Stable liberalism (CV = 0.75) but extreme restrictionism volatility (CV = 17.8)

\textbf{Interpretation:} Established Americans maintain liberal immigration values but show extreme variation in enforcement preferences.

\begin{figure}[H]
    \centering
    \includegraphics[width=0.95\textwidth]{../../outputs/CURRENT_2025_08_09_FIGURES_gold_standard/volatility_by_generation.png}
    \caption{\textbf{Attitude Stability Patterns:} Different generations show distinct volatility profiles, with first generation most responsive to policy changes.}
    \label{fig:volatility}
\end{figure}

\section{Theoretical Revolution: Beyond Linear Assimilation}

\subsection{What Traditional Theory Predicted}
\begin{itemize}
    \item \textbf{Linear progression:} 1st Gen (restrictionist) → 2nd Gen (moderate) → 3rd+ Gen (liberal)
    \item \textbf{Stable directionality:} Each generation more ``American'' than the last
    \item \textbf{Gradual convergence:} Attitudes slowly approach mainstream American positions
    \item \textbf{Cultural transmission:} Parents pass restrictionist attitudes to children
\end{itemize}

\subsection{What Our Data Actually Shows}
\begin{itemize}
    \item \keyfinding{\textbf{Non-linear trajectories:} Three distinct response mechanisms}
    \item \keyfinding{\textbf{Role reversal:} 1st generation becomes most liberal by 2022}
    \item \keyfinding{\textbf{Counter-cyclical behavior:} 2nd generation opposes 1st generation}
    \item \keyfinding{\textbf{Policy responsiveness:} Attitudes actively respond to political environment}
\end{itemize}

\subsection{New Theoretical Framework: Multi-Track Response Model}

\textbf{Track-Specific Predictions:}
\begin{enumerate}
    \item \textbf{Reactive Integration Track:} Recent immigrants liberalize through positive integration experiences and policy threat response
    \item \textbf{Strategic Distancing Track:} Second generation distances from immigrant identity during periods of immigration salience
    \item \textbf{Settled Americanism Track:} Established populations maintain stable liberal baselines with minimal policy responsiveness
\end{enumerate}

\section{Sociologically Surprising Findings}

\subsection{1. The Integration Liberalization Paradox}
\surprise{\textbf{Surprise:}} The longer first-generation immigrants live in America, the MORE pro-immigration they become.

\textbf{Traditional Expectation:} Immigrants adopt mainstream (restrictionist) American attitudes\\
\textbf{Reality:} Integration experiences create stronger pro-immigration positions

\subsection{2. Second-Generation Strategic Distancing}
\surprise{\textbf{Surprise:}} The most ``American'' generation (2nd) becomes most restrictionist during immigration debates.

\textbf{Traditional Expectation:} Second generation serves as cultural bridge\\
\textbf{Reality:} Second generation actively distances from immigrant identity

\subsection{3. Policy Environment Anticipation Effects}
\surprise{\textbf{Surprise:}} Attitude changes precede policy implementation by 1-2 years.

\textbf{Traditional Expectation:} Policies shape attitudes\\
\textbf{Reality:} Anticipated policies shape attitudes before implementation

\subsection{4. The Third+ Generation Bipolar Pattern}
\surprise{\textbf{Surprise:}} Most established Americans show extreme volatility in restrictionism despite stable liberalism.

\textbf{Traditional Expectation:} Long-term Americans have stable, moderate attitudes\\
\textbf{Reality:} Complex, multidimensional attitude structure with domain-specific volatility

\section{Implications for Electoral Politics}

\subsection{The ``Latino Vote'' Myth}
\begin{itemize}
    \compactdesc{Media Narrative}{Uniform rightward shift among Latinos}
    \compactdesc{Reality Check}{Generational divergence creates internal polarization}
    \compactdesc{Electoral Strategy}{Generation-specific messaging required}
    \compactdesc{Temporal Effects}{4-year electoral cycles miss 20-year attitude trends}
\end{itemize}

\subsection{Coalition Instability Implications}
\begin{itemize}
    \item \textbf{Democratic Assumptions:} Latino solidarity on immigration may be weakening
    \item \textbf{Republican Opportunities:} Second-generation strategic distancing creates openings
    \item \textbf{Future Projections:} Generational replacement effects will reshape baseline attitudes
    \item \textbf{Issue Framing:} Immigration debates activate generational cleavages
\end{itemize}

\begin{figure}[H]
    \centering
    \includegraphics[width=0.95\textwidth]{../../outputs/CURRENT_2025_08_09_FIGURES_gold_standard/generation_trends_2002_2022.png}
    \caption{\textbf{Generational Divergence:} Each generation follows distinct trajectories that challenge linear assimilation models and create electoral unpredictability.}
    \label{fig:generation_trends}
\end{figure}

\section{Policy Design Implications}

\subsection{Generation-Specific Policy Frames}
\begin{enumerate}
    \item \textbf{First Generation Messaging:}
        \begin{itemize}
            \item Emphasize integration success stories
            \item Highlight anti-discrimination protections
            \item Connect current policies to personal experiences
        \end{itemize}
    
    \item \textbf{Second Generation Messaging:}
        \begin{itemize}
            \item Avoid explicit immigrant identity framing
            \item Emphasize American values and fairness
            \item Focus on economic and security considerations
        \end{itemize}
    
    \item \textbf{Third+ Generation Messaging:}
        \begin{itemize}
            \item Use standard liberal humanitarian appeals
            \item Emphasize historical American immigrant tradition
            \item Connect to broader civil rights frameworks
        \end{itemize}
\end{enumerate}

\subsection{Temporal Sensitivity in Policy Timing}
\begin{itemize}
    \compactdesc{Anticipatory Effects}{Policy announcements create immediate attitude shifts}
    \compactdesc{Delayed Reactions}{First generation responses lag 2-3 years}
    \compactdesc{Consolidation Periods}{Post-policy attitude stabilization takes time}
    \compactdesc{Strategic Windows}{Optimal timing for policy initiatives varies by generation}
\end{itemize}

\section{Methodological Contributions}

\subsection{Analytical Innovations}
\begin{itemize}
    \compactdesc{NOFE Approach}{``No Fixed Effects'' methodology captures linear trends effectively}
    \compactdesc{Multi-Index System}{Three-dimensional attitude measurement reveals complexity}
    \compactdesc{Fine-Grained Temporal Analysis}{Year-by-year examination reveals inflection points}
    \compactdesc{Volatility Assessment}{Coefficient of variation analysis shows stability patterns}
\end{itemize}

\subsection{Data Quality Validation}
\begin{itemize}
    \compactdesc{Coverage Assessment}{60\%+ coverage on core indices provides robust inference}
    \compactdesc{Temporal Resolution}{20-year span captures multiple policy cycles}
    \compactdesc{Cross-Validation}{Multiple measures confirm generational patterns}
    \compactdesc{Robustness Checks}{Joinpoint analysis validates linear trend assumptions}
\end{itemize}

\begin{figure}[H]
    \centering
    \includegraphics[width=0.8\textwidth]{../../outputs/CURRENT_2025_08_09_FIGURES_gold_standard/data_coverage_matrix.png}
    \caption{\textbf{Data Quality Matrix:} Variable availability across survey years demonstrates robust coverage for core theoretical questions.}
    \label{fig:coverage}
\end{figure}

\section{Future Research Agenda}

\subsection{Immediate Priorities}
\begin{enumerate}
    \item \textbf{Legalization Collapse Investigation:} Determine if 0\% support reflects measurement artifact or genuine attitude shift
    \item \textbf{Mechanism Testing:} Survey experiments on identity threat and policy response
    \item \textbf{Individual-Level Analysis:} What predicts within-generation variation?
    \item \textbf{Comparative Framework:} How do Latino patterns compare to other immigrant groups?
\end{enumerate}

\subsection{Theoretical Development}
\begin{enumerate}
    \item \textbf{Multi-Track Model Refinement:} Formal theoretical statement with testable predictions
    \item \textbf{Policy Responsiveness Theory:} How do different generations process political information?
    \item \textbf{Identity Formation Mechanisms:} What drives second-generation strategic distancing?
    \item \textbf{Integration Liberalization Theory:} Why does integration increase pro-immigration attitudes?
\end{enumerate}

\subsection{Applied Research}
\begin{enumerate}
    \item \textbf{Electoral Behavior Studies:} Do attitude differences predict voting patterns?
    \item \textbf{Policy Experiment Design:} Can messaging bridge generational divides?
    \item \textbf{Geographic Variation:} How do local contexts modify generational patterns?
    \item \textbf{Longitudinal Individual Tracking:} How do attitudes evolve within persons over time?
\end{enumerate}

\section{Conclusion: The Generational Revolution}

\subsection{Paradigm Shift}
Our analysis reveals that \keyfinding{``immigrants against immigrants'' reflects complex generational response mechanisms, not simple attitude convergence}. The 2021-2022 role reversal fundamentally challenges 100+ years of assimilation theory.

\subsection{Sociological Significance}
\begin{itemize}
    \item \textbf{Theoretical Impact:} Invalidates linear assimilation models
    \item \textbf{Methodological Lesson:} Aggregate stability can mask revolutionary change
    \item \textbf{Political Relevance:} Generational cleavages reshape electoral coalitions
    \item \textbf{Policy Importance:} Generation-specific approaches required
\end{itemize}

\subsection{The Multi-Track Future}
As Latino populations continue to grow and diversify, understanding these three distinct generational ``tracks'' becomes essential for:
\begin{itemize}
    \item Predicting electoral behavior
    \item Designing effective immigration policies  
    \item Understanding American political assimilation
    \item Anticipating future demographic-political trends
\end{itemize}

\textbf{Bottom Line:} The story of Latino immigration attitudes is not one of linear assimilation, but of \textbf{complex, non-linear generational divergence} that creates new forms of political identity and coalition instability in 21st century America.

\section{Technical Appendix}

\subsection{Enhanced Data Quality Standards}
\begin{itemize}
    \compactdesc{Temporal Coverage}{20 years, 13 waves, 4 administrations}
    \compactdesc{Sample Power}{30,869 observations with valid generation labels}
    \compactdesc{Statistical Approach}{Weighted OLS (NOFE) with robust standard errors}
    \compactdesc{Coverage Validation}{Core indices 59-63\% coverage, adequate for inference}
    \compactdesc{Fine-Grained Analysis}{Year-by-year examination reveals critical inflection points}
    \compactdesc{Cross-Validation}{Multiple indices confirm generational patterns}
\end{itemize}

\subsection{Gold Standard Analysis Scripts}
\begin{itemize}
    \item Fine-Grained Analysis: \texttt{CURRENT\_2025\_08\_09\_ANALYSIS\_fine\_grained\_patterns.R}
    \item Gold Standard Visuals: \texttt{CURRENT\_2025\_08\_09\_UPDATE\_gold\_standard\_visuals.R}
    \item Three Indices Plot: \texttt{CURRENT\_2025\_08\_09\_ANALYSIS\_three\_indices\_plot.R}
    \item Bug-Fixed Scripts: \texttt{CURRENT\_2025\_08\_09\_CREATE\_FIXED\_supplementary\_gold\_standard.R}
\end{itemize}

\subsection{Data Archive}
\begin{itemize}
    \item Detailed yearly data: \texttt{CURRENT\_2025\_08\_09\_liberalism\_yearly\_detailed.csv}
    \item Generational gaps: \texttt{CURRENT\_2025\_08\_09\_generational\_gaps\_detailed.csv}
    \item Volatility analysis: \texttt{CURRENT\_2025\_08\_09\_volatility\_analysis\_detailed.csv}
    \item Period effects: \texttt{CURRENT\_2025\_08\_09\_period\_effects\_detailed.csv}
\end{itemize}

\vspace{1em}
\noindent\rule{\textwidth}{0.2pt}

\begin{center}
\textbf{Analysis Version:} Gold Standard v2025.08.09 | \textbf{Revolutionary Findings Edition}\\
\textbf{Theoretical Contribution:} Multi-Track Generational Response Model\\
\textbf{Date:} August 9, 2025
\end{center}

\end{document}
